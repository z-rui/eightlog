% 纸张: A5

\pdfpageheight 148mm
\pdfpagewidth 105mm

\hsize=105mm
\vsize=148mm
\advance \hsize by -1in \hoffset -.5in
\advance \vsize by -1.5in \voffset -.45in

% 字体相关
\input typezh
\catcode`@=11
\baselineskip=15pt
\setbox\strutbox=\hbox{\vrule height11.5pt depth3.5pt width0pt}
\font\sixteensong="华文中宋:letterspace=-15" at 16pt
\font\eightsong="宋体" at 8pt
\font\fourteenhei="黑体" at 14pt
\font\tenhei="黑体" at 10pt
\font\eighthei="黑体" at 8pt
\font\eightbf=cmbx8
\font\sixbf=cmbx6
\font\eightrm=cmr8
\font\sixrm=cmr6
\font\eighti=cmmi8
\font\sixi=cmmi6
\font\eightsy=cmsy8
\font\sixsy=cmsy6
\font\tenib=cmmib10
\font\eightib=cmmib8
\def\bf{\fam\bffam\zhdualfont\tenhei\tenbf}
\newfam\ibfam \textfont\ibfam=\tenib
\def\mib{\fam\ibfam\tenib}
\def\bflg{\mathop{\bf lg}}

% 8磅字用于排版数表,改编自 manmac.tex
\def\eightpoint{\def\rm{\fam0\zhdualfont\eightsong\eightrm}%
  \textfont0=\eightrm \scriptfont0=\sixrm \scriptscriptfont0=\fiverm
  \textfont1=\eighti \scriptfont1=\sixi \scriptscriptfont1=\fivei
  \textfont2=\eightsy \scriptfont2=\sixsy \scriptscriptfont2=\fivesy
  \textfont3=\tenex \scriptfont3=\tenex \scriptscriptfont3=\tenex
  \textfont\ibfam=\eightib
  %\def\it{\fam\itfam\eightit}% 注释掉的字体本文档中用不到,省略
  %\textfont\itfam=\eightit
  %\def\sl{\fam\slfam\eightsl}%
  %\textfont\slfam=\eightsl
  \def\bf{\fam\bffam\zhdualfont\eighthei\eightbf}%
  \textfont\bffam=\eightbf \scriptfont\bffam=\sixbf
  % \scriptscriptfont\bffam=\fivebf
  %\def\tt{\fam\ttfam\eighttt}%
  %\textfont\ttfam=\eighttt
  %\tt \ttglue=.5em plus.25em minus.15em
  \normalbaselineskip=9pt
  \setbox\strutbox=\hbox{\vrule height7pt depth2pt width\z@}%
  \normalbaselines\rm}

% 章节标题
\outer\def\title#1\par{\vbox to .7in
  {\vss\centerline{\zhsinglefont\fourteenhei#1}}\bigskip}

% 页脚
\footline={\ifodd\pageno \hss\stylefolio \else \stylefolio\hss \fi}
\def\stylefolio{\rm $\sim$\ \folio\ $\sim$}

% 封面
\outer\def\cover#1\par{\shipout\vbox to\vsize{
\vskip \z@ plus .382fil
\centerline{\zhsinglefont\sixteensong#1}
\vskip \z@ plus .618fil
}}

% 语录
\outer\def\quotation#1\par{
  \vbox to \vsize\bgroup\baselineskip=0pt\vss
  \def\simfill{\xleaders\hbox{$\sim$}\hfill}% 横框线
  \def\wrfill{\xleaders\hbox to 0pt{\hss$\wr$\hss}\vfill}% 竖框线
  \halign to \hsize\bgroup&##\cr
  \simfill{\zhsinglefont\sixteensong #1}\simfill\cr
  \noalign{\vskip-5pt}% 微调框线间距
  \valign\bgroup&##\cr\wrfill\cr\vbox\bgroup
  \zhsinglefont\fourteenhei\zhnormalskips
  \baselineskip=19pt \parskip\baselineskip \parindent=28pt
  \narrower \null}
\outer\def\endquotation{
  \vskip\parskip \egroup% 结束内部 \vbox
  \cr\wrfill\cr\egroup% 结束 \valign
  \cr\noalign{\vskip-2pt}% 微调框线间距
  \simfill\cr\egroup% 结束 \halign
  \vss\egroup}% 结束外部 \vbox

% 带上划线的数字1和5
\def\1{\overbar1}
\def\5{\overbar5}
\def\overbar#1{\hbox{$\m@th\overline#1$}}
% 数表中左侧加星号
\def\*{\llap{*\kern-\p@}}

% 表格工具
% 横线
\def\lefttab{\begintab\leftheader}
\def\righttab{\begintab\rightheader}
% 表头(行)
\def\leftheader{$\mib N$\hfil&&
  \bf 0\hfil&&\bf 1\hfil&&\bf 2\hfil&&\bf 3\hfil&&\bf 4\hfil\cr}
\def\rightheader{%
  \bf 5\hfil&&\bf 6\hfil&&\bf 7\hfil&&\bf 8\hfil&&\bf 9\hfil&&
  $\mib N$\hfil\cr}
% 表2
\def\begintab#1{\vbox\bgroup
  \eightpoint\offinterlineskip
  \def\blankline{&&&&&&&&&&\cr}
  \def\smallgap{\omit&height 2\p@&\omit&&\omit&&\omit&&\omit&&\omit\cr}
  \spaceskip=.2em \tabskip\centering
  \halign to \hsize\bgroup &\strut\hfil##&\vrule##\cr
  \toprule #1\midrule}
% 表2、表3通用
\def\endtab{\botrule \egroup \egroup % 关闭 \valign 和 \vbox
  \eject}% 不用 \vfill,输出例程 \taboutput 会填充页面
\def\toprule{\noalign{\hrule height\p@}\smallgap}
\def\midrule{\smallgap\noalign{\hrule \vskip\p@ \hrule}\smallgap}
\def\botrule{\smallgap\noalign{\hrule height\p@}}
% 表3
\def\longheader{$\mib N$\hfil&&$\bflg \mib N$\hfil&&
                $\mib N$\hfil&&$\bflg \mib N$\hfil\cr}
\def\longmidrule{\smallgap\noalign{\copy\z@}\multispan\thr@@&height\p@\cr
                          \noalign{\copy\z@}\smallgap}
\def\longtab{\vbox\bgroup
  \eightpoint\offinterlineskip
  \def\blankline{&&&&&&\cr}
  \def\smallgap{\omit&height2\p@&&&\omit&&\cr}
  \setbox\z@=\hbox to \hsize{\hrulefill\hskip\p@\hrulefill}
  \tabskip\centering
  \halign to \hsize\bgroup
  &\strut\hfil##&\vrule##&\hfil##&\vrule##\kern\p@\vrule\cr
  \toprule \longheader \longmidrule}
  
% 表格输出例程
\newtoks\tabtitle % 表格标题(放在偶数页)
\newtoks\N \newtoks\lgN % N 和 lgN 的范围(放在奇数页)
\def\taboutput{\setbox\@cclv\vbox{
  \ifodd\pageno \rightline{\strut\bf $\mib N$\quad\the\N\qquad}% N 范围
  \else         \leftline{\strut\bf\qquad\the\tabtitle}\fi% 表格标题
  \smallskip\unvbox\@cclv
  \ifodd\pageno \nointerlineskip\smallskip
    \rightline{\strut\bf $\bflg \mib N$\quad\the\lgN\qquad}\fi% lgN 范围
  \vss}
  \plainoutput}

% 《用法说明》工具
\outer\def\step#1\par{{\bf#1}\par\nobreak} % 计算XXX:
\outer\def\Ex#1 #2\par{\smallskip{\bf [例~#1]}\quad#2\par\nobreak}
\outer\def\Sol{{\bf 解}\quad}
% 列竖式用对齐环境
\def\aralign#1{\vcenter{\offinterlineskip\lineskip=\jot
  \ialign{&\mathstrut$\displaystyle{{}##{}}$\hfil&
  $\displaystyle##$\hfil\crcr
  #1\crcr}}}
% 竖式中画横线(前#1列)
\def\hline#1{\multispan#1\hrulefill\cr}
% 带框公式
\def\boxed#1{\vrule\vcenter{\hrule\vskip 4pt\hbox{\hskip 4pt$\displaystyle#1$
  \hskip 4pt}\vskip 4pt\hrule}\vrule}

\catcode`@=12
